\documentclass[UTF8]{article}
\usepackage[zihao = -4]{ctex}
\usepackage{fontspec}
\usepackage{titlesec}
\usepackage[hidelinks]{hyperref}
\usepackage{enumitem}
\usepackage{graphicx}
\usepackage{float}
\usepackage{siunitx}
\usepackage{subcaption}
\usepackage{tabularx}
\usepackage{amsmath}
\usepackage{array}

\title{\textbf{力学与机械波实验}}
\author{}
\date{}

\titleformat{\section}[hang]{\Large\bfseries}{\thesection}{1em}{}
\titleformat{\subsection}[hang]{\zihao{-4}\bfseries}{\thesubsection}{1em}{}

\begin{document}
    \thispagestyle{empty}
    \maketitle
    \newpage
    \thispagestyle{empty}
    \tableofcontents
    \newpage


    \section{运动守恒定律实验}
        \subsection{实验目的}
            \noindent\hspace{2em}(1)理解动能、重力势能、机械能守恒;掌握估测小球动能和重力势能的方法;验证机械能守恒定律。
            
            \noindent\hspace{2em}(2)理解动量、动量守恒;掌握估测小球速度和动量的方法,验证动量守恒定律。
        \subsection{实验原理}
            \noindent\hspace{2em}以双线牛顿单摆为观测对象,通过测量小球的速度估算动能,通过测量其高度估算重力势能。速度和高度的测量通过钢尺+手机
            拍摄视频的方式进行数据采集,利用开源软件Tracker进行视频分析和小球速度、高度估计。

            \noindent\hspace{2em}以双单摆为观测对象,通过速度测量估算小球的动量;通过两个小球的对心碰撞验证动量守恒定律。速度的测量方法同上。
        \subsection{实验仪器}
            \noindent\hspace{2em}双线牛顿单摆、双线牛顿双摆、钢卷尺、手机、计算机。
        \subsection{实验步骤}
            \begin{enumerate}[left=2em, label=(\arabic*)]
                \item 放置双线单摆。
                \item 将手机尽量垂直于摆动平面,在一定距离处固定好并开启录像功能。
                \item 将钢尺平行于摆动平面并尽量靠近单摆放置。
                \item 将小球置于某一高度并放开,令其进行自由摆动。
                \item 用手机拍摄小球摆动视频,利用Tracker软件分析计算小球的位置,速度。
                \item 利用公式
                $$ E_k = \frac{1}{2}mv^2, E_p = mgh, E_m = E_k + E_p $$                
                计算小球在不同时刻的动能、重力势能,验证机械能守恒定律。
                \item 改变小球初始高度进行多次实验。
                \item 添加另一个双线单摆,形成双单摆。
                \item 将小球A置于某一高度并放开,令其摆动并与小球B碰撞。
                \item 用手机拍摄小球碰撞视频,利用Tracker软件分析计算两个小球碰撞前后的速度。
                \item 利用公式
                $$ p = mv, \Delta p = p_{\text{before}} - p_{\text{after}} $$
                计算小球在碰撞前后的动量变化,验证动量守恒定律。
                \item 改变小球初始高度、质量比进行多次实验。
            \end{enumerate}
        \subsection{实验数据}
            \begin{enumerate}
                \item 验证机械能守恒(忽略空气阻力):
                    \begin{enumerate}[left=2em, label=\arabic*)]
                        \item 初始高度$h=\SI{6}{cm}$:
                            \begin{table}[H]
                                \centering
                                \begin{tabularx}{\textwidth}{
                                    c |
                                    >{\centering\arraybackslash}X
                                }
                                \hline
                                \begin{minipage}{0.45\textwidth}
                                    \centering
                                    \includegraphics[width=\textwidth]{E:/files/working/latex/物理实验报告/image/60.png}
                                \end{minipage}
                                &
                                \begin{minipage}{\linewidth}
                                    \begin{tabular}{
                                        || >{\centering\arraybackslash}p{0.248\linewidth}
                                        | >{\centering\arraybackslash}p{0.248\linewidth}
                                        | >{\centering\arraybackslash}p{0.248\linewidth}
                                        ||
                                    }
                                        \hline
                                        动能(J) & 重力势能(J) & 机械能(J)\\ \hline
                                        5.792e-5 & 3.081e-3 & 3.139e-3 \\ \hline
                                        1.444e-3 & 1.541e-3 & 2.985e-3 \\ \hline
                                        1.697e-3 & 1.154e-3 & 2.851e-3\\ \hline
                                    \end{tabular}                                    
                                \end{minipage}

                                \\ \hline
                                \end{tabularx}
                                \caption{初始高度$h=\SI{6}{cm}$}
                                \label{form:60}
                            \end{table}
                            在忽略空气阻力的情况下,可以近似认为机械能守恒。
                        \item 初始高度$h=\SI{7.5}{cm}$:
                            \begin{table}[H]
                                \centering
                                \begin{tabularx}{\textwidth}{
                                    c |
                                    >{\centering\arraybackslash}X
                                }
                                \hline
                                \begin{minipage}{0.45\textwidth}
                                    \centering
                                    \includegraphics[width=\textwidth, height=0.25\textheight]{E:/files/working/latex/物理实验报告/image/75.png}
                                \end{minipage}
                                &
                                \begin{minipage}{\linewidth}
                                    \begin{tabular}{
                                        || >{\centering\arraybackslash}p{0.248\linewidth}
                                        | >{\centering\arraybackslash}p{0.248\linewidth}
                                        | >{\centering\arraybackslash}p{0.248\linewidth}
                                        ||
                                    }
                                        \hline
                                        动能(J) & 重力势能(J) & 机械能(J)\\ \hline
                                        5.506e-3 & 4.856e-4 & 5.991e-3 \\ \hline
                                        9.096e-4 & 5.072e-3 & 5.981e-3 \\ \hline
                                        3.289e-3 & 2.644e-3 & 5.932e-3\\ \hline
                                    \end{tabular}                                    
                                \end{minipage}

                                \\ \hline
                                \end{tabularx}
                                \caption{初始高度$h=\SI{7.5}{cm}$}
                                \label{form:75}
                            \end{table}
                            在忽略空气阻力的情况下,可以近似认为机械能守恒。

                        
                        \item 初始高度$h=\SI{9.5}{cm}$:
                        \begin{table}[H]
                            \centering
                            \begin{tabularx}{\textwidth}{
                                c |
                                >{\centering\arraybackslash}X
                            }
                            \hline
                            \begin{minipage}{0.45\textwidth}
                                \centering
                                \includegraphics[width=\textwidth]{E:/files/working/latex/物理实验报告/image/95.png}
                            \end{minipage}
                            &
                            \begin{minipage}{\linewidth}
                                \begin{tabular}{
                                    || >{\centering\arraybackslash}p{0.248\linewidth}
                                    | >{\centering\arraybackslash}p{0.248\linewidth}
                                    | >{\centering\arraybackslash}p{0.248\linewidth}
                                    ||
                                }
                                    \hline
                                    动能(J) & 重力势能(J) & 机械能(J)\\ \hline
                                    1.177e-2 & 1.006e-2 & 2.183e-2 \\ \hline
                                    7.144e-3 & 1.446e-2 & 2.160e-2 \\ \hline
                                    1.605e-2 & 1.055e-2 & 2.120e-2\\ \hline
                                \end{tabular}                                    
                            \end{minipage}

                            \\ \hline
                            \end{tabularx}
                            \caption{初始高度$h=\SI{9.5}{cm}$}
                            \label{form:95}
                        \end{table}
                        在忽略空气阻力的情况下,可以认为机械能守恒。
                    \end{enumerate}
                \item 验证动量守恒:
                    \begin{enumerate}[left=2em, label=\arabic*)]
                        \item 质量比1.728,初始高度6cm
                            \begin{figure}[H]
                                \centering
                                \begin{subfigure}{0.45\textwidth}
                                    \centering
                                    \includegraphics[width=\textwidth]{E:/files/working/latex/物理实验报告/image/1A.png}
                                    \caption{小球A}
                                    \label{fig:1A}
                                \end{subfigure}
                                \hfill
                                \begin{subfigure}{0.45\textwidth}
                                    \centering
                                    \includegraphics[width=\textwidth]{E:/files/working/latex/物理实验报告/image/1B.png}
                                    \caption{小球B}
                                    \label{fig:1B}
                                \end{subfigure}
                                \caption{质量比1.728,初始高度6cm}
                                \label{fig:1.728-6}
                            \end{figure}
                            \vfill
                            \begin{table}[H]
                                \centering
                                \begin{tabularx}{\textwidth}{
                                    || >{\centering\arraybackslash}X
                                    | >{\centering\arraybackslash}X
                                    | >{\centering\arraybackslash}X
                                    | >{\centering\arraybackslash}X
                                    ||
                                }
                                    \hline
                                    碰撞次数 & 小球A速度(m/s) & 小球B速度(m/s) & 总动量(kg$\cdot$m/s)\\ \hline
                                    第一次前: & 3.65e-1 & 0 & 1.09e-2 \\ \hline
                                    第一次后: & 2.86e-1 & 2.39e-1 & 1.27e-2 \\ \hline
                                    第二次前: & 2.88e-1 & 3.16e-1 & 1.41e-2\\ \hline
                                    第二次后: & 2.23e-1 & 4.56e-1 & 1.46e-2 \\ \hline
                                    第三次前: & -2.19e-1 & -3.32e-1 & -1.23e-2 \\ \hline
                                    第三次后: & -3.14e-1 & -1.84e-1 & -1.26e-3\\ \hline
                                    
                                \end{tabularx}
                                \caption{质量比1.728,初始高度6cm\protect\footnotemark}
                                \label{form:1.728-6}
                                
                            \end{table}
                            \footnotetext{由于手机拍摄产生的丢帧和不良截取问题,部分数据并没有采用最接近碰撞时刻的数据。
                            碰撞次数也是便于观察的第一,第二,第三次。}                                
                        \item 质量比3.835,初始高度6cm
                            \begin{figure}[H]
                                \centering
                                \begin{subfigure}{0.45\textwidth}
                                    \centering
                                    \includegraphics[width=\textwidth]{E:/files/working/latex/物理实验报告/image/2A.png}
                                    \caption{小球A}
                                    \label{fig:2A}
                                \end{subfigure}
                                \hfill
                                \begin{subfigure}{0.45\textwidth}
                                    \centering
                                    \includegraphics[width=\textwidth]{E:/files/working/latex/物理实验报告/image/2C.png}
                                    \caption{小球C}
                                    \label{fig:C}
                                    
                                \end{subfigure}
                                \caption{质量比3.835,初始高度6cm}
                                \label{fig:3.835-6}
                            \end{figure}
                            \vfill
                            \begin{table}[H]
                                \centering
                                \begin{tabularx}{\textwidth}{
                                    || >{\centering\arraybackslash}X
                                    | >{\centering\arraybackslash}X
                                    | >{\centering\arraybackslash}X
                                    | >{\centering\arraybackslash}X
                                    ||
                                }
                                    \hline
                                    碰撞次数 & 小球A速度(m/s) & 小球C速度(m/s) & 总动量(kg$\cdot$m/s)\\ \hline
                                    第一次前: & 3.07e-1 & 0 & 9.20e-3 \\ \hline
                                    第一次后: & 2.46e-1 & 2.24e-1 & 9.22e-3 \\ \hline
                                    第二次前: & -1.64e-1 & -3.96e-1 & -8.06e-3\\ \hline
                                    第二次后: & -2.72e-1 & -3.19e-2 & -8.42e-3 \\ \hline
                                    第三次前: & -1.71e-1 & -4.53e-1 & -8.84e-3 \\ \hline
                                    第三次后: & -2.02e-1 & -1.50e-1 & -7.30e-3\\ \hline
                                    
                                \end{tabularx}
                                \caption{质量比3.835,初始高度6cm}
                                \label{form:3.835-6}
                            \end{table}
                        \item 质量比1.728,初始高度7.5cm
                            \begin{figure}[H]
                                \centering
                                \begin{subfigure}{0.45\textwidth}
                                    \centering
                                    \includegraphics[width=\textwidth]{E:/files/working/latex/物理实验报告/image/3A.png}
                                    \caption{小球A}
                                    \label{fig:3A}
                                \end{subfigure}
                                \hfill
                                \begin{subfigure}{0.45\textwidth}
                                    \centering
                                    \includegraphics[width=\textwidth]{E:/files/working/latex/物理实验报告/image/3B.png}
                                    \caption{小球B}
                                    \label{fig:3B}
                                \end{subfigure}
                                \caption{质量比1.728,初始高度7.5cm}
                                \label{fig:1.728-7.5}
                            \end{figure}
                            \vfill
                            \begin{table}[H]
                                \centering
                                \begin{tabularx}{\textwidth}{
                                    || >{\centering\arraybackslash}X
                                    | >{\centering\arraybackslash}X
                                    | >{\centering\arraybackslash}X
                                    | >{\centering\arraybackslash}X
                                    ||
                                }
                                    \hline
                                    碰撞次数 & 小球A速度(m/s) & 小球B速度(m/s) & 总动量(kg$\cdot$m/s)\\ \hline
                                    第一次前: & 1.19e-1 & -1.73e-2 & 3.27e-3 \\ \hline
                                    第一次后: & -1.45e-2 & 1.67e-1 & 2.46e-3 \\ \hline
                                    第二次前: & -1.64e-1 & -3.96e-1 & -8.06e-3\\ \hline
                                    第二次后: & -2.72e-1 & -3.19e-2 & -8.42e-3 \\ \hline
                                    第三次前: & -1.71e-1 & -4.53e-1 & -8.84e-3 \\ \hline
                                    第三次后: & -2.02e-1 & -1.50e-1 & -7.30e-3\\ \hline
                                
                                    
                                \end{tabularx}
                                \caption{质量比1.728,初始高度7.5cm}
                                \label{form:1.728-7.5}
                            \end{table}

                 
                        \item 质量比3.835,初始高度7.5cm
                            \begin{figure}[H]
                                \centering
                                \begin{subfigure}{0.45\textwidth}
                                    \centering
                                    \includegraphics[width=\textwidth]{E:/files/working/latex/物理实验报告/image/4A.png}
                                    \caption{小球A}
                                    \label{fig:4A}
                                \end{subfigure}
                                \hfill
                                \begin{subfigure}{0.45\textwidth}
                                    \centering
                                    \includegraphics[width=\textwidth]{E:/files/working/latex/物理实验报告/image/4C.png}
                                    \caption{小球C}
                                    \label{fig:4C}
                                \end{subfigure}
                                \caption{质量比3.835,初始高度7.5cm}
                                \label{fig:3.835-7.5}
                            \end{figure}
                            \vfill
                            \begin{table}[H]
                                \centering
                                \begin{tabularx}{\textwidth}{
                                    || >{\centering\arraybackslash}X
                                    | >{\centering\arraybackslash}X
                                    | >{\centering\arraybackslash}X
                                    | >{\centering\arraybackslash}X
                                    ||
                                }
                                    \hline
                                    碰撞次数 & 小球A速度(m/s) & 小球C速度(m/s) & 总动量(kg$\cdot$m/s)\\ \hline
                                    第一次前: & 3.77e-1 & 0 & 1.13e-2 \\ \hline
                                    第一次后: & 3.82e-1 & 1.69e-1 & 1.28e-2 \\ \hline
                                    第二次前: & -9.98e-2 & -2.80e-1 & -5.21e-3\\ \hline
                                    第二次后: & -1.94e-1 & -5.21e-2 & -6.25e-3 \\ \hline
                                    第三次前: & 5.33e-2 & 8.85e-2 & 2. 30e-3 \\ \hline
                                    第三次后: & -6.37e-2 & 1.49e-1 & -7.31e-4\\ \hline
                                    
                                \end{tabularx}
                                \caption{质量比3.835,初始高度7.5cm}
                                \label{form:3.835-7.5}
                            \end{table}
                    \end{enumerate}
            \end{enumerate}
        \subsection{实验结论}
            \noindent\hspace{2em}
                \begin{enumerate}[left=2em, label=\arabic*)]
                    \item 在忽略空气阻力的情况下,可以认为机械能守恒。
                    $$ E_{\text{before}} = E_{\text{after}} $$
                    \item 在误差允许范围内,可以认为动量守恒。
                    $$ m_1v_1 + m_2v_2 = m_1v_1' + m_2v_2' $$
                \end{enumerate}
    \section{机械振动与机械波实验}
        \subsection{实验目的}
            \begin{enumerate}[left=2em, label=(\arabic*)]
                \item 理解简谐振动,机械波的概念。
                \item 掌握估测小球运动轨迹和运动状态的方法;观测分析单个小球的简谐振动和多个小球形成的机械波。
            \end{enumerate}
        \subsection{实验原理}
            \noindent\hspace{2em}以牛顿单摆为观测对象,观测分析小球的简谐振动;以不同摆长的牛顿摆为观测对象,观测分析多个小球形成的机械波。

            \noindent\hspace{2em}速度和位置的测量通过钢尺+手机拍视频的方式进行数据采集,利用开源软件Tracker进行视频分析和小球速度、位置估计。
        \subsection{实验仪器}
            \noindent\hspace{2em}双线牛顿单摆、不同摆长的双线牛顿摆、钢卷尺、手机、计算机。
        \subsection{实验步骤}
            \begin{enumerate}[left=2em, label=(\arabic*)]
                \item 放置双线单摆。
                \item 将手机尽量垂直于小球摆动平面,在一定距离处固定好并开启录像功能。
                \item 将钢尺平行于摆动平面并尽量靠近单摆放置。
                \item 将小球置于某一高度并放开,令其进行自由摆动。
                \item 用手机拍摄小球摆动视频,利用Tracker软件分析计算小球的位置、速度。
                \item 改变小球初始高度、摆长进行多次实验。
                \item 放置不同长度的多个单摆,将手机摄像头从侧上方对准多单摆。
                \item 用长直尺将多摆中的所有小球侧推至同一角度后放开,令其进行自由摆动。
                \item 用手机拍摄小球摆动视频,利用Tracker软件分析计算每个小球的位置、速度变化。进而分析多个小球形成的机械波的周期,频率变化。
                \item 分析机械波的周期、频率与摆长的关系。
            \end{enumerate}
        \subsection{实验数据}
            \begin{enumerate}[left=2em, label=\arabic*.]
                \item 单个小球分析:
                    \begin{figure}[H]
                        \centering
                        \includegraphics[width=\textwidth]{E:/files/working/latex/物理实验报告/image/75.png}
                        \caption{7.5cm高度}
                        \label{fig:7.5cm}
                    \end{figure}

                    此情况
                    $${v}_{max}=\SI{0.929}{m/s}$$
                    小球的最大位移为
                    $$x_{max}=\SI{0.132}{m}$$

                    \begin{figure}[H]
                        \centering
                        \includegraphics[height=0.5\textheight]{E:/files/working/latex/物理实验报告/image/95.png}
                        \caption{9cm高度}
                        \label{fig:9.5cm}
                    \end{figure}


                    此情况
                    $${v}_{max}=\SI{0.964}{m/s}$$
                    小球的最大位移为
                    $$x_{max}=\SI{0.144}{m}$$

                    \begin{figure}[H]
                        \centering
                        \includegraphics[height=0.5\textheight]{E:/files/working/latex/物理实验报告/image/60.png}
                        \caption{6cm高度}
                        \label{fig:6cm}
                    \end{figure}
    
                    此情况
                    $${v}_{max}=\SI{0.452}{m/s}$$
                    小球的最大位移为
                    $$x_{max}=\SI{0.100}{m}$$

                    数据如下:
                    \begin{table}[H]
                        \centering
                        \begin{tabularx}{0.8\textwidth}{
                        || >{\centering\arraybackslash}X
                        | >{\centering\arraybackslash}X
                        | >{\centering\arraybackslash}X
                        ||
                    }
                        \hline
                        初始高度(cm) & 最大位移(m) & 最大速度(m/s) \\ \hline
                        6 & 0.100 & 0.452 \\ \hline
                        7.5 & 0.132 & 0.929 \\ \hline
                        9.5 & 0.144 & 0.964 \\ \hline
                        \end{tabularx}
                        \caption{不同高度速度位移的变化}
                        \label{form1}
                    \end{table}
                    由此得小球的速度与初始高度正相关,与摆长无关;
                    小球的位置与初始高度正相关,与摆长成负相关。
                \item 多个小球分析:
                    \begin{figure}[H]
                        \centering
                        \begin{subfigure}[h]{0.45\textwidth}
                            \centering
                            \includegraphics[width=\textwidth]{E:/files/working/latex/物理实验报告/image/A.png}
                            \caption{球A}
                            \label{1}
                        \end{subfigure}
                        \begin{subfigure}[h]{0.45\textwidth}
                            \centering
                            \includegraphics[width=\textwidth]{E:/files/working/latex/物理实验报告/image/B.png}
                            \caption{球B}
                            \label{2}
                        \end{subfigure}
                        \hfill
                        \begin{subfigure}[h]{0.45\textwidth}
                            \centering
                            \includegraphics[width=\textwidth]{E:/files/working/latex/物理实验报告/image/C.png}
                            \caption{球C}
                            \label{3}
                        \end{subfigure}
                        \begin{subfigure}[h]{0.45\textwidth}
                            \centering
                            \includegraphics[width=\textwidth]{E:/files/working/latex/物理实验报告/image/D.png}
                            \caption{球D}
                            \label{4}
                        \end{subfigure}
                        \hfill
                        \begin{subfigure}[h]{0.45\textwidth}
                            \centering
                            \includegraphics[width=\textwidth]{E:/files/working/latex/物理实验报告/image/E.png}
                            \caption{球E}
                            \label{5}
                        \end{subfigure}
                    \end{figure}
                    数据如下:
                    \begin{table}[H]
                        \centering
                        \begin{tabularx}{0.8\textwidth}{
                            || >{\centering\arraybackslash}X
                            | >{\centering\arraybackslash}X
                            | >{\centering\arraybackslash}X
                            ||
                        }
                            \hline
                            小球名称 & 摆长(m) & 周期(s) \\ \hline
                            A & 0.185 & 0.866 \\ \hline
                            B & 0.205 & 0.900 \\ \hline
                            C & 0.210 & 0.917 \\ \hline
                            D & 0.215 & 0.922 \\ \hline
                            E & 0.220 & 0.933 \\ \hline
                        \end{tabularx}
                        \caption{机械波周期与摆长关系}
                        \label{form2}
                    \end{table}
                    可得单摆的周期随摆长变长而增大,频率随摆长变长而减小。
            \end{enumerate}
        
        \subsection{实验结论}
            \begin{enumerate}[left=2em, label=\arabic*)]
                \item 小球在$x$方向上的运动可以近似认为是简谐运动。
                \item 小球速度与初始高度正相关,与摆长无关。
                \item 小球位置与初始高度正相关,与摆长负相关。
                \item 小球机械波周期与摆长正相关,频率相反。
            \end{enumerate}
            以上实验结果符合忽略空气阻力的位置公式:
            \begin{equation*}
                \begin{cases}
                    \ l(1-cos(\theta)) = h \\
                    \ lsin(\theta) = x
                \end{cases}
            \end{equation*}
            以及小球速度公式(忽略空气阻力):
            \begin{equation*}
                v = \sqrt{2gh}
            \end{equation*}
            多个小球分析则符合单摆周期公式:
            \begin{equation*}
                T = 2\pi \sqrt{\frac{l}{g}}
            \end{equation*}
\end{document}